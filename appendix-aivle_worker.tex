\chapter{aiVLE Worker Design}
\label{appendix:aivle-worker}

\section{Comparison of Mainstream Security Solutions}
\label{as:comparison-of-security-solutions}

There are several ``asterisks'' to the claims listed in Table~\ref{tab:security-solutions}, which we are going to address in detail in this appendix section. 

First, by ``rootless'' we meant no root privilege is required \emph{after} the initial setup. All solutions listed, including Firejail, requires root privilege to install. However, using it afterwards does not require root privilege.

Second, why in Docker and Podman is considered to be ``rootless'', but we still say that they ``prevent the GPU from being shared by any other container with root access''? It is because both Docker and Podman have a ``rootless mode'', which allows application to run without root access, however it also forces the GPU to run on rootless mode. Take Docker as an example, Docker Engine 20.10 introduced rootless Docker daemon\footnote{\href{https://docs.docker.com/engine/security/rootless/}{https://docs.docker.com/engine/security/rootless/}}. But just like Podman, in rootless mode to make NVidia container runtime work you still need to set no-cgroups = true\footnote{\href{https://www.redhat.com/en/blog/how-use-gpus-containers-bare-metal-rhel-8}{https://www.redhat.com/en/blog/how-use-gpus-containers-bare-metal-rhel-8}}, therefore breaks root access from containers\footnote{\href{https://github.com/NVIDIA/nvidia-container-runtime/issues/85}{https://github.com/NVIDIA/nvidia-container-runtime/issues/85}}.

Third, the ``level of isolation'' of Firejail is labelled as ``medium'' in the table. In fact, Firejail security entirely relies on how strict the security profile is – by default there is no restriction at all. The recommended approach according to Firejail documentation\footnote{\href{https://firejail.wordpress.com/documentation-2/building-custom-profiles/}{https://firejail.wordpress.com/documentation-2/building-custom-profiles/}} is to start with a strictest possible profile, then gradually white-listing features until the application inside the sandbox works fine. Since the full lockdown provided by Firejail is very strict, we consider it as a ``secure-enough'' alternative.

Lastly, VM is labelled to have no GPU support in the table. However, VM does support \textbf{exclusive} GPU access by technologies such as \href{https://docs.fedoraproject.org/en-US/Fedora/13/html/Virtualization_Guide/chap-Virtualization-PCI_passthrough.html}{PCI passthrough}. Exclusivity is the keyword here: we require the solution to work on shared instances, therefore we ignore such infeasible solutions.
