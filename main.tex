\documentclass[fyp]{socreport}
\usepackage[hidelinks]{hyperref}
\usepackage{graphicx} % to insert images
\usepackage{float}
%%% code block setups
\usepackage[newfloat]{minted}
\usepackage{xcolor} % to access the named colour LightGray
\definecolor{LightGray}{gray}{0.9}
\usepackage{caption}
\newenvironment{code}{\captionsetup{type=listing}}{}
\SetupFloatingEnvironment{listing}{name=Code}
%%% code block setups
\usepackage{fullpage}
\usepackage{multirow}
\usepackage{longtable}
\usepackage{amsmath}
% \usepackage{tocbibind}
\interfootnotelinepenalty=10000 %% Completely prevent breaking of footnotes

\begin{document}
\pagenumbering{roman}
\title{Competition Platform for AI Tasks}
\author{Tan Yuanhong}
\projyear{2021/22}
\projnumber{H247080}
\advisor{Dr. Akshay Narayan, Prof. Leong Tze Yun}
\deliverables{
	\item Report: 1 Volume
% 	\item Source Code: 6 Git Repository
}
\maketitle
\begin{abstract}
Entering the second decade of the 21st century, machine learning and artificial intelligence is the new must-have for computer science education. However, while the data-oriented tasks like classification and regression have well-adopted platforms such as Kaggle, simulation-oriented tasks that are most common for reinforcement learning (RL) algorithms are yet to have a feature-complete online judging solution. This project aims to build a complete solution that is extensible, scalable, and easy-to-use. It covers four critical components of judging RL algorithms: a simulation environment framework, an auto-grading framework, a scalable and secure sandbox for arbitrary code execution, and finally a modern web application as the entry point. Besides a comprehensive documentation and many quality-of-life new features, this project greatly improves the performance over existing internal platform (aiVLE 1.0) with a task queue system that not only distributes tasks fairly, but also dynamically adapts to the load of each worker node thanks to its resource awareness. Experimental results show that the new system provides equal task distribution among worker nodes and nearly three times of resource utilization improvement over aiVLE 1.0 ($>90\%$ vs $\approx 30\%$).

\begin{project-nature}
	Implementation, Experimentation, Simulation
\end{project-nature}
\begin{keywords}
    \item Artificial Intelligence
	\item Machine Learning
\end{keywords}
\begin{implement}
	Ubuntu 20.04, Firejail 0.9.68, Python 3.8, Django 4.0, Celery 5.2
\end{implement}

\end{abstract}

\begin{acknowledgement}
I would like to thank my friends and family for their support during the hard times of the COVID-19 pandemic, especially my significant other for her patiently listening to my annoyingly frequent mentions of this project.\\
I would also like to thank my advisors Dr. Narayan and Prof. Leong. They offered an unimaginable amount of guidance to this final year project with great kindness.\\
Without them, this project would not have been possible.\\
\end{acknowledgement}

\listoffigures 
\listoftables
\tableofcontents

\chapter{Introduction}

With the increasing popularity in both artificial intelligence (AI) research and industry application, more and more institutions are considering AI education as an integral part of their computer science or even general education curriculum. However, in spite the algorithmic nature of AI education, there is surprisingly few platforms and tools to host AI competitions/assignments and automatically grade the submissions. Unlike traditional algorithm and data structure courses that have many online judges available (to name a few, \href{https://codeforces.com/}{Codeforces}, \href{https://open.kattis.com/}{Kattis}, \href{https://onlinejudge.org/}{UVa Online Judge}), AI courses often rely on arbitrary grading scripts or in-house solutions to host their assignments/contests. There are well-known data science competition platforms like \href{https://www.kaggle.com/}{Kaggle} that are suitable for prediction/classification tasks, but the area of reinforcement learning (RL) tasks\footnote{To be precise, here tasks mean environments that are typically used for RL algorithms. The differentiating characteristic of such tasks is that they are interactive (in contrast to comparing the output against the ground truth like what Kaggle and other traditional OJs do).} remains surprisingly untouched. Therefore, this project tries to fill the gap of evaluating algorithms that solve interactive tasks. We aim to provide a complete solution from building RL\footnote{Technically the evaluation model we use is suitable for almost all AI/ML tasks (including classification and regression), but we will focus on RL first since we already have Kaggle for classification/regression tasks.} environments, writing test cases, to eventually hosting these tasks on a massively scalable platform.

\section{Background}
\subsection{Reinforcement Learning Tasks}
Reinforcement learning (RL) is a branch of machine learning that maps observations (possibly with history as well) to actions such that certain metrics are maximized~\cite{sutton-barto}. The metrics which agents try to maximize are called a reward or reinforcement, thus the name “reinforcement” learning. In a broader sense, any AI task is RL task: we can generalize any input as the observation, any output as the action and any target as the reward. But for the sake of simplicity, in this report, by RL task we mean sequential decision problems (SDP), where the outcome of agent’s decision depends on a sequence of decisions~\cite{russell}.

A sequential decision problem is fully described by state space, action space, transition model and a reward function~\cite{russell}. A transition model is a function that maps (current state, history states, action) to a probability of reaching any possible state. If the transition model does not depend on history state, we call it Markovian as the probability only depends on the current state instead of the history of all earlier states. A reward function maps current state\footnote{Reward function may also depend on action and next state, but one that only depends on the current state is the most common variant, and this difference is not fundamental.} to a numerical reward.

The solution to a sequential decision problem is called a policy. It is a function that maps current state (and sometimes history states) to an action. We always assume that the domain of policy function is the entire observation space so that the agent knows what action to take under any possible circumstance. The goal of RL algorithms is to come up with such policy functions that maximize their expected reward under different settings (e.g., unknown underlying transition model, partial observability, etc.).

From the definition of sequential decision problems, we can summarize the following execution loop of any typical SDP:

\begin{figure}[htp]
    \centering
    \includegraphics{images/obr-loop.png}
    \caption{Action-observation-reward loop}
    \label{fig:obr-loop}
\end{figure}

In other words, the environment implements the transition model $P\left(s_j\middle| s_i,a\right)$ and reward function $r\left(s_i\right)$ and returns the next observation $o_j$ and reward $r_j$ to the agent, after which the agent takes another step of action $a_{j+1}$ based on $\left[o_j,o_{j-1},\cdots,o_0\right]$ and $\left[r_j,r_{j-1},\cdots,r_0\right]$. Note that $o_i$ and $s_i$ may or may not be the same depending on whether the environment is fully observable.

\subsection{Reinforcement Learning Environment Frameworks And Collections}
RL research, like many computer science topics, requires concretizing the theory with implementation and experiments as they are equally important. That is where RL environment frameworks and collections like the Arcade Learning Environment (ALE)~\cite{ale} and OpenAI Gym~\cite{openai-gym} come into play. Such an effort is incredibly important for the rapid growth of the RL community: it not only saves researchers a lot of emulation development time, but also provides benchmark environments to evaluate RL tasks similar to how CIFAR-10~\cite{cifar-10} provides benchmark datasets in image classification. A typical OpenAI Gym agent code is structured as below:

\begin{code}
\begin{minted}[
frame=lines,
framesep=2mm,
baselinestretch=1.2,
bgcolor=LightGray,
fontsize=\footnotesize,
linenos
]{python}
import gym

env = gym.make('CartPole-v0')
for i_episode in range(100):
    observation = env.reset()
    for t in range(100):
        action = decide(observation, reward)
        observation, reward, done, info = env.step(action)
env.close()
\end{minted}
\captionof{listing}{OpenAI Gym agent example}
\label{code:gym-agent}
\end{code}

Besides collecting many classical environments as a benchmark package, an important contribution of OpenAI Gym is its widely accepted application programming interface (API). This common interface proves to be sufficient for not only hundreds of self-contained environments included in OpenAI Gym, but also thousands more third-party environments. When it comes to turning retro video games to RL environments, Gym dominates the game by eventually having ALE adopt its API and merge into Gym. Moreover, as if conquering Atari2600 is not enough, Gym Retro \cite{gym-retro} adds even more emulated systems such as Nintendo Game Boy and Sega Genesis to Gym. Therefore, it is safe to say that following this interface gives us enough flexibility to implement most of the desired environments.

From the source code of OpenAI Gym, a compatible environment needs to implement the methods and attributes shown in Figure~\ref{fig:gym-class}:

\begin{figure}[htp]
    \centering
    \includegraphics{images/gym-class.png}
    \caption{OpenAI Gym Class Diagram}
    \label{fig:gym-class}
\end{figure}

Authors of OpenAI Gym listed multi-agent as a future direction when they published the paper in 2016. However, as of 2021, there is still no official support for multi-agent tasks. There are some third-party environments that support multi-agent tasks, like ma-gym \cite{magym}. These implementations usually convert multi-agent observation, reward, and action into a vector, length of which is equal to the number of participating agents in the environment.

\section{Related Work}

\subsection{Online Judging System}
Online judges first emerged to address the problem of automatically grading programming assignments conveniently, fairly and securely \cite{RN4}. Without any manual intervention once the system is configured properly, online judge systems need to handle 1) receiving submissions, 2) compiling and executing source codes, 3) comparing outputs against the correct answers, 4) logging the result. For fairness, RAM and CPU usage will be monitored and limited. Usually, there is a hard time limit after which any submission will be forced to terminate. For security, execution of arbitrary submission needs to be contained in a safe environment with heavily restricted privilege levels such that only necessary operations are allowed, any other attempt (e.g., accessing the network, read/write on filesystem, etc.) will be blocked and preferably reported to the system administrator. 

Among all the challenges in building a successful online judge, security is probably the most technically difficult aspect. Some online judges were used in an exam setting where students had to use dedicated terminals for access, and a system administrator is involved in monitoring the security issues \cite{10.1145/384267.305835}. For automated security measures, there are generally two approaches: 1) source code or binary scanning, and 2) operating system level access control. 

Source code or binary scanning has no runtime performance penalty, but it requires an exhaustive check on all possible malicious tokens/system calls. It is unrealistic to have a perfectly secure source code scanner and every new supported language\footnote{A “new” language does not have to be a completely new one. If new keywords or new compiler version is introduced, there could possibly be new security implications.} requiring a new set of scanning rules makes this approach non-sustainable \cite{RN4}. Besides, it is impossible to scan the binary for programs written in a dynamic programming language such as Python as they execute inside an interpreter.

Access control based on operating system kernel security features is much more foolproof and widely applicable. Such techniques are also called OS-level virtualization because from the perspective of programs inside a ``namespace'', it can only see partial filesystem and devices that are assigned to this “namespace”. Since the introduction of user namespaces in Linux kernel 3.8, containerization technology has become increasingly popular to provide isolation between processes while being able to share nearly all hardware resources with minimal performance overhead \cite{RN16}. 

With the rapid growth of cloud and distributed computing, there are also novel online judges that attempt to adopt these cutting-edge technologies. A distributed online judge architecture tries to address the problem of scalability: an online judge needs to execute student submissions, and with more students and courses, computation required grows linearly. It is difficult to scale up (making computers more powerful) but easy to scale out (having more computers) especially in recent years with the emergence of cloud computing \cite{RN17}. Evaluating AI tasks has an even stronger demand for horizontal scalability as neural networks or sophisticated searching algorithms are notoriously computationally heavy. Notable implementations of distributed online judge include MetaOJ~\cite{metaoj}, which separates data storage, web application and judgers into three separate components that can be deployed to the cloud independently (Figure~\ref{fig:metaoj}).

\begin{figure}[H]
    \centering
    \includegraphics{images/metaoj.png}
    \caption{MetaOJ architecture}
    \label{fig:metaoj}
\end{figure}

\subsection{ML/RL Benchmark Platforms}
In contrast to competition platforms on the education side, researchers are more interested in a benchmark platform that provides a collection of tasks to compare different RL algorithms. Such benchmarks are crucial for reproducible and verifiable research in RL. According to~\cite{RN20}, an interpretable and reproducible experiment consists of four parts: algorithm, parameters, evaluation method and dataset. For a benchmark, evaluation method and dataset are fixed while contributors will compare different algorithms and parameters. In the context of RL benchmark, this means packaging the simulation environment and evaluation metrics behind a simple set of common interfaces for both training and testing.

However, as machine learning models (along with their training processes) get much more sophisticated, reproducibility is affected by not only those four points – random seeds, version of external libraries and even specific model of CPU/GPU could greatly affect the result of the same code \cite{RN21}. Therefore, according to “Ten Simple Rules for Reproducible Computational Research” by \cite{RN22}, for an RL benchmark platform to be reproducible, it should also: 1) have strict versioning\footnote{Meaning “any” change to the implementation of either environment or evaluation metrics should be reflected in a different version number. This also includes changes to the version of external dependencies.} of environments and evaluation metrics; 2) use deterministic methods whenever possible and fix random seeds throughout the process; 3) have a record of all raw data.

Considering the criteria listed above, OpenAI Gym (Brockman et al., 2016) is possibly one of the closest to an ideal benchmark: it explicitly embraces strict versioning, has an interface to fix random seeds, and includes many well-accepted tasks. Unfortunately, it only handles the simulation side of RL benchmark while leaving the evaluation to the users. For a successful ML benchmark platform like Kaggle, it not only provides dataset (environment in RL) but also enforces common evaluation criteria on all submissions so that algorithms are truly comparable.

\subsection{AI Competition Platforms}
\label{ch:literature-review-related-work-ai-competition-platforms}
In both sections Introduction and Project Objective, we mentioned that there are few RL task competition platforms. The available ones have pain points that make them unviable for many use cases. In this section, we compare aiVLE 1.0, an internally developed platform for CS4246 as a baseline of improvement for this project.

aiVLE 1.0 addresses the need for evaluating an OpenAI Gym agent in a GPU-accelerated environment. The architecture of aiVLE 1.0 is illustrated in Figure~\ref{fig:aivle-1-arch}:

\begin{figure}[H]
    \centering
    \includegraphics{images/aivle_1_arch.png}
    \caption{Architecture of aiVLE 1.0}
    \label{fig:aivle-1-arch}
\end{figure}

Due to urgency and time constraints, the author of aiVLE 1.0 prioritized “making it work” over “doing it right”, which leaves us four pain points\footnote{The list here is not exhaustive: more limitations and considerations will be discussed in Implementation Progress section, along with solutions to these pain points.} that impact its extensibility:

\begin{enumerate}
    \item Lack of documentation. For example, 1) deployment documentation for the runner instances (programs that evaluate student submissions), 2) architecture overview of each component and the entire system, and 3) inline documentation such as function usage, class definition. Without these documentation, new developers and users will take a much longer time to adopt and contribute to the system. 
    \item Need more coverage of software engineering best practices such as proper level of abstraction, and avoiding lengthy functions/files for interpretability, etc.
    \item No frontend-backend separation. Navigating and understanding the codebase is difficult as there is no clear boundary between frontend and backend code. Moreover, using JavaScript for interactivity would make the codebase look even more convoluted.
    \item No multi-agent support.
\end{enumerate}

For scalability, lack of proper task queue makes it difficult to scale aiVLE 1.0 to more machines: its workers get evaluation tasks by requesting server for the latest submissions \textit{periodically}. Worker-side polling has risks for traffic congestion and race condition, and the consequent lack of load balancing further hurts aiVLE 1.0’s potential to scale well with more workers – more details are covered in Section~\ref{ch:aivle-web_highly-available-task-queue}.
\chapter{Project Objective}
\label{ch:project-objective}
\section{Problem Statement}
\label{s:project-objective-problem-statement}
The goal of this project is to provide a complete solution for RL algorithm evaluation that is extensible, scalable, and easy-to-use. It consists of four tightly integrated components (collectively called aiVLE 2.0):

\begin{enumerate}
    \item aiVLE\footnote{aiVLE is name for the grading system currently used in CS4246, more details about the old aiVLE (aiVLE 1.0) are covered in Section~\ref{ch:literature-review-related-work-ai-competition-platforms}} Gym: An OpenAI Gym \cite{openai-gym} compatible RL environment framework with agent-environment separation and official support for multi-agent tasks.
    \item aiVLE Grader: An auto grading framework for aiVLE Gym tasks.
    \item aiVLE Worker: A security sandbox to execute arbitrary code submissions safely in a controlled environment, and a massively scalable worker client for evaluation.
    \item aiVLE Web: A web application for hosting RL competitions.
\end{enumerate}

\section{Motivation}
\label{s:project-objective-motivation}
Courses like CS4246, which teach AI, and are different from traditional programming/algorithm courses, need a system to automate the process of collecting and evaluating programming assignments on AI tasks. aiVLE (AI Virtual Learning Environment) is the inspiration and foundation of this project – it is an RL task evaluation system built by the CS4246 teaching team since 2019. We will call the old aiVLE as aiVLE 1.0 henceforth\footnote{Do note that aiVLE 1.0 refers to the system as a whole –- it does not refer to any specific component (e.g., web, runner, runner-kit).}. aiVLE 1.0 provides instant feedback on programming assignments that require varied computational power such as GPU or other specialized processing units. However, aiVLE 1.0 has many pain points. For example, it lacks extensibility (e.g., does not support multi-agent task), scalability (e.g., no concurrency safety for many workers), and documentation for more courses to make use of the platform. As for another similar platform called Botzone~\cite{botzone}, although it’s built for external users, limitations like no GPU support and no common interface make it infeasible for many tasks.

Secondly, an open source RL competition platform with good documentation and software engineering best practices has potential impact beyond education purposes. On the one hand, such a platform could also be useful for benchmarking AI algorithms – the consistency required for grading assignments is perfect for comparing research as well. On the other hand, examples like AlphaGo~\cite{alphago} proves the effectiveness of combining supervised learning from past match data with unsupervised RL – match history collected on the platform could be useful for training models for corresponding tasks. 

Lastly, there is demand for such a platform. Both CS3243 and CS4246 have assignments on RL algorithms. And even for many non-RL chapters such as brute force or informed search, satisfiability problems, the techniques could also be used to solve RL tasks. It is a safe bet that such a platform will benefit many AI courses (including CS2109S and CS3244) by using it alongside platforms like Kaggle to cover most AI algorithm evaluation scenarios.

\section{Roadmap}
\label{s:project-objective-roadmap}
There are three stages planned for this project. During the first semester, we delivered satisfactory results for the first two stages. During the winter break, we conducted \hyperref[ch:deployment-and-testing]{real-world deployment and testing} to evaluate the feasibility of advancing the project to the third stage. During the second semester, we focus on fixing problems found during the test, and delivering features as mentioned in the third stage.

\subsection{Stage 1: Framework}
Before developing the web platform for hosting the competitions, we first need frameworks for 1) creating environment and 2) evaluating agent’s performance – \hyperref[ch:aivle-gym]{aiVLE Gym} and \hyperref[ch:aivle-grader]{Grader} respectively. These frameworks are independent from the web platform, therefore could be used separately for other education or research purposes as well.
Besides platform-independent frameworks, a security sandbox, \hyperref[ch:aivle-worker]{aiVLE Worker}, is also necessary for hosting competitions. These three frameworks will be implemented in the mentioned order as the latter have dependency on the former.

\subsection{Stage 2: Platform}
With the frameworks ready, the second stage is to have \hyperref[ch:aivle-web]{a web platform} that achieves basic online judge functionality, which includes but is not limited to: 1) user registration and authentication, 2) creating/joining courses, 3) creating/modifying/submitting to tasks, 4) showing the evaluation result. The target of this stage is to have a \emph{minimum viable product} that CS4246 can use to host single-agent assignments with GPU-accelerated evaluation.

\subsection{Stage 3: Advanced Solution}
Stage 1 and 2 provide a solid foundation for further extension and exploration, but primarily focus on delivering features that are necessary (thus \textbf{minimally} viable) for CS4246 teaching. Stage 3 aims to further extend the use case of our web platform. First, we will improve the documentation of the source code and use cases, which is crucial for the maintainability and upgradability of this project. Second, we will add advanced features that are so-called nice-to-have\footnote{There are also many other nice-to-have features that did not appear in this report, like more user-friendly frontend and course administration tools. To keep the report relatively concise, features that do get covered are either technically challenging or ``interesting''.} (e.g., Sections \ref{ss:aivle-worker-resource-awareness}, \ref{ss:aivle-web-load-balancing}) but are achievable within the time constraint. Lastly, we may explore the application of this project in terms of teaching multi-agent algorithms, human-in-the-loop computing and even try building a test bed for RL algorithms and host the benchmark on the same platform for researchers\footnote{``A unified testbed for AI teaching and research'' by Ho Hol Yin (Project ID H247060)}.
\chapter{Design and Implementation}
\label{ch:design-and-impl}
Following the nomenclature defined in Section~\ref{ch:literature-review-related-work-ai-competition-platforms}, we will call the new system aiVLE 2.0 henceforth. Do note that aiVLE 2.0 also refers to the new system architecture as a whole~–~every sub-project has an independent versioning that does not necessarily share the major revision\footnote{This is because components like aiVLE Gym and aiVLE Grader does not exist in aiVLE 1.0.} number 2. 

As mentioned in Section~\ref{s:project-objective-problem-statement}, there are four components in aiVLE 2.0: aiVLE Gym, Grader, Worker and Web. As a high-level overview, Figure~\ref{fig:architecture-overview} shows the relationship between the components: the system can be roughly divided into client-side (where users write and run the agents) and server-side (where the platform hosts the competition and evaluates the submissions). On the client side, we have \textbf{aiVLE Gym} (Section~\ref{ch:aivle-gym}) responsible for simulation. \textbf{aiVLE Grader} (Section~\ref{ch:aivle-grader}) calls the agent implementation to interact with the environment and record the simulations. All these arbitrary code execution happens inside the security sandbox of \textbf{aiVLE Worker} (Section~\ref{ch:aivle-worker}), ensuring fair resource allocation and secure code execution. On the server side, we have a worker cluster consisting of many worker nodes, communicating with \textbf{aiVLE Web} (Section~\ref{ch:aivle-web}) in an orchestrated manner. This cluster is designed to be fault tolerant and massively scalable. And of course, aiVLE Web also provides a frontend that you can access from a browser.

\begin{figure}[H]
    \centering
    \includegraphics[width=0.9\textwidth]{images/architecture-overview.png}
    \caption{aiVLE 2.0 Architecture Overview}
    \label{fig:architecture-overview}
\end{figure}

In this chapter, we will discuss about the design and implementation of all these four components in detail, in the order of Gym, Grader, Worker and finally aiVLE Web. In fact, this is also the chronological order of development: the later components may have dependencies on earlier ones, and the earlier components generally\footnote{Strictly speaking, aiVLE Worker does not depend on aiVLE Web, but when it is used as a worker client on top of a security sandbox, although it will startup normally, without aiVLE Web to push evaluation jobs into the task queue, the worker client is meaningless. So we used ``generally'' here just to be precise.} do not depend on the later ones. For example, aiVLE Gym can be used as general-purpose multi-agent environment framework outside of the aiVLE 2.0 platform.

\section{aiVLE Gym - Separating Agents from Environment}
\label{ch:aivle-gym}
I have released a stable version of aiVLE Gym with all planned features implemented. \href{https://github.com/edu-ai/aivle-gym}{The source} consists of $\sim$1K lines of code, including several example environments and full documentation. The package is published to PyPI  (\href{https://test.pypi.org/project/aivle-gym}{https://test.pypi.org/project/aivle-gym}) so that aiVLE Gym can be installed and imported like any other Python package.

\subsection{Motivation}
aiVLE Gym makes multi-agent competition possible by separating agents from the environment. In a two-agent OpenAI Gym task, we write agent code as shown in Code~\ref{code:two-agent-example}:

\begin{code}
\begin{minted}[frame=lines,framesep=2mm,baselinestretch=1.2,bgcolor=LightGray,fontsize=\footnotesize,linenos]{python}
env = gym.make("PongDuel-v0") # Two-player Ping Pong game

for ep_i in range(100):
    done_n = [False for _ in range(env.n_agents)]
    ep_reward = 0
    obs_n = env.reset()
    env.render()
    while not all(done_n):
        action_0 = decide_0(obs_n[0], reward_n[0])
        action_1 = decide_1(obs_n[1], reward_n[1])
        action_n = [action_0, action_1]
        obs_n, reward_n, done_n, info = env.step(action_n)
        ep_reward += sum(reward_n)
        env.render()
    print('Episode #{} Reward: {}'.format(ep_i, ep_reward))

env.close()
\end{minted}
\captionof{listing}{OpenAI Gym agent example}
\label{code:two-agent-example}
\end{code}

Note that in a multi-agent scenario, \texttt{observation}, \texttt{reward} and \texttt{done} are all vectors - each element corresponds to one of the agents. Similarly, when you call \texttt{env.step()}, you should provide \texttt{action}s for every agent in this simulation. Such design is acceptable when we perform these multi-agent experiments offline. However, in a competition setting, when it comes to multi-agent tasks, you cannot make decisions for your opponent agents. Therefore, separating agents from the environment simulation is necessary. From the perspective of each agent, it is just like a single-agent environment – the only difference is that the environment is affected by actions taken by other agents as well. Figure~\ref{fig:opanai-gym-multi-arch} and Figure~\ref{fig:aivle-gym-multi-arch} show the architectural differences between \emph{OpenAI} Gym and \emph{aiVLE} Gym (each colored box represents a separate process; solid arrows represent inter-process or network communication):
\begin{figure}[H]
    \centering
    \includegraphics{images/opanai-gym-multi-arch.png}
    \caption{Multi-agent Architecture for OpenAI Gym}
    \label{fig:opanai-gym-multi-arch}
\end{figure}
\begin{figure}[H]
    \centering
    \includegraphics{images/aivle-gym-multi-arch.png}
    \caption{Multi-agent Architecture for aiVLE Gym}
    \label{fig:aivle-gym-multi-arch}
\end{figure}

\subsection{Design Goal}
Unless mentioned otherwise, all design goals in this chapter are achieved in the released implementation. 

The design goal of aiVLE Gym is to keep full compatibility with OpenAI Gym on the agent side. On the environment side, it lets you convert from existing OpenAI Gym environments with little adaptation. More specifically:

In single-agent case, on the agent side, traditional environment (simulation happens within agent process) and aiVLE environment (simulation happens outside of agent process) should be interchangeable. On the environment side, author can reuse existing OpenAI Gym compatible environment by implementing a serializer that serializes \texttt{action}, \texttt{observation}, and \texttt{info} to JSON compatible objects.

In multi-agent case, on the agent side, the APIs behaves just like normal single-agent OpenAI Gym environment. On the environment side, author can reuse existing ma-gym environment~\cite{magym} by implementing a serializer along with several metadata fields. 

\subsection{Agent-Environment Communication}
\label{ss:agent-env-communication}
Since agents and environment are separated, there needs to be an inter-process communication channel between them. aiVLE Gym uses a lightweight yet high-performance messaging library ZeroMQ~\cite{zeromq}, which has comprehensive support for many synchronous and asynchronous messaging patterns that are essential to this project. There are two primary challenges for multi-agent tasks when it comes to agent-environment communication:
\begin{enumerate}
    \item The judge should receive and respond to requests asynchronously - it needs to wait for all agents' actions before stepping forward in the environment, then decide what observations/rewards to respond to each of the agents.
    \item Certain operations (e.g., resetting the simulation) must be performed strictly once for each episode, but since each agent will initialize the episode on their own, judge-side will unavoidably receive multiple requests.
\end{enumerate}

To summarize, the judge-side environment needs to implement a ``barrier'' synchronization mechanism that not only realizes synchronous rendezvous of agent requests, but also performs additional tasks upon the ``first-comer'' and ``last-leaver''. 

Therefore, we propose the deterministic finite automaton (DFA) as shown in Figure~\ref{fig:aivle-gym-multi-dfa}:
\begin{figure}[H]
    \centering
    \includegraphics{images/aivle-gym-multi-dfa.png}
    \caption{aiVLE Gym Multi-agent Environment Communication Automaton}
    \label{fig:aivle-gym-multi-dfa}
\end{figure}

This DFA is key to keeping complicated communication details transparent to both agent and environment logic – both can write common synchronous code while the framework deals with the underlying asynchronous logic. Details of this DFA and messaging patterns involved can be found in the Appendix~\ref{as:aivle-gym_dfa}.

\section{aiVLE Grader - Evaluating Agents Using Test Suites}
\label{ch:aivle-grader}
I have released a stable version of aiVLE Grader with support for 1) OpenAI Gym, 2) aiVLE Gym single-agent, and 3) aiVLE Gym multi-agent environment. The codebase consists of $\sim$600 lines of framework code and $\sim$300 lines of example test suites for all three supported use cases. Similar to aiVLE Gym, the package is released to PyPI  (\href{https://test.pypi.org/project/aivle-grader/}{https://test.pypi.org/project/aivle-grader}) for in-production usage.

\subsection{Design Goal}
Unlike competitive programming style problems or machine learning prediction tasks, evaluating RL agents is much more complicated than comparing students’ output against a standard answer. Fortunately, with the common programming interfaces provided by OpenAI/aiVLE Gym, on top of them we may create a framework that standardizes/modularizes the initialization, execution, and conclusion of RL agent evaluation. The ultimate goal of this framework, when writing a grader for agents in any OpenAI/aiVLE Gym environment, is to:
\begin{enumerate}
    \item Make the built-in components so complete that for most use cases using built-in ones would be sufficient.
    \item Make each component self-contained without complicated inter-dependencies (i.e., following the single responsibility principle) when writing a custom component.
\end{enumerate}

\subsection{Key Abstractions}
There are three key abstractions to aiVLE Grader: \textit{agent}, \textit{evaluator}, and \textit{test case} as summarized in Figure~\ref{fig:aivle-grader-class}.

\begin{figure}[H]
    \centering
    \includegraphics{images/aivle-grader-class.png}
    \caption{Class Diagram for aiVLE Grader}
    \label{fig:aivle-grader-class}
\end{figure}

\textit{Agent} only has two methods: \texttt{reset} to reset internal states, \texttt{step} to return an action from provided observation. It is flexible enough to allow agents to memorize the history, whilst restrictive enough to prohibit agents from modifying the inner-workings of the environment.

\textit{Evaluator} records the entire execution process and produces a score when the session terminates. It utilizes the common pattern of most RL tasks (see Figure~\ref{fig:obr-loop}): each session consists of many episodes, and each episode consists of many concrete steps. By inserting hook functions to these critical points, an evaluator practically records everything about the evaluation session. 

\textit{Test case} is a bootstrap for evaluation sessions. It wraps \textit{agent}, \textit{environment}, and \textit{evaluator} along with necessary initialization parameters into an object with one simple \texttt{evaluate} method. It also offloads certain chore (e.g., time limit) away from the user.

\section{aiVLE Worker - Secure and Scalable Grading Client}
\label{ch:aivle-worker}
I have released the second\footnote{Relative to the first stable version released during the first semester.} stable version of aiVLE Worker. The most significant upgrade from the second release is its resource awareness (see Section~\ref{ss:aivle-worker-resource-awareness}). It is validated on both CPU-only machines and GPU nodes with CUDA drivers (GTX 1050Ti for CUDA 10, RTX 3070 for CUDA 11). Moreover, it has been deployed and tested on several GPU nodes of the SoC Compute Cluster (see Section~\ref{s:deployment}). The codebase consists of $\sim$700 LoC, also with examples and detailed documentation. 

Unlike aiVLE Gym and Grader that are packages that are meant to be imported in other scripts, aiVLE Worker is a self-contained client that is runnable out-of-the-box. Thus aiVLE Worker is published to PyPI (\href{https://test.pypi.org/project/aivle-worker}{https://test.pypi.org/project/aivle-worker})  as a command-line tool. Users may install aiVLE Worker from PyPI and use it like any regular program (as shown in Figure~\ref{fig:aivle-worker-cli}).

\begin{figure}[H]
    \centering
    \includegraphics[width=\textwidth]{images/aivle-worker-cli.png}
    \caption{aiVLE Worker Command-line Interface}
    \label{fig:aivle-worker-cli}
\end{figure}

\subsection{Design Goal}
\label{ss:aivle-worker-design-goal}
For security, the implementation is expected to achieve:
\begin{enumerate}
    \item File access restriction: agent program should have no access to directories that may contain sensitive data (e.g., private keys), and should have read only access to files necessary for its execution (e.g., Python binary, dependencies, agent, and environment source code).
    \item Network restriction: agent program should have no access to the Internet. Otherwise, (1) it may benefit from extra computation resources; (2) it may send out confidential runtime details (e.g., configuration of simulation environment) that allow users to fine-tune their program; both of which make the competition unfair.
    \item Resource limit: agent program should be terminated and reported if it exceeds RAM limit, VRAM limit or time limit.
\end{enumerate}

For scalability, in addition to the task queue system that manages all worker nodes and distributes evaluation jobs fairly and efficiently to the workers (see Section~\ref{ch:aivle-web_highly-available-task-queue}), on the worker side, a client that requires little permission and setup would be extremely helpful. It benefits horizontal scalability because 1) adding new worker nodes is easier, 2) more machines can be used as worker nodes (e.g., shared GPU servers, programming lab PCs). Thus, what we expect from this solution are:
\begin{enumerate}
    \item Managed concurrency: it should be able to run as many evaluation jobs concurrently as the hardware resource permits, while being able to detect and terminate processes that consume excessive RAM/VRAM. For more details please refer to Section~\ref{ss:aivle-worker-resource-awareness}.
    \item Minimal permission requirement: if we have access to run the submission locally, the environment should be able to operate as a worker node (i.e., sudo is not required).
    \item Minimal dependency requirement: any Linux machine with Python should be able to run the worker client.
    \item Moderate overhead: compared to traditional OJ, aiVLE can trade some overhead (both startup and runtime) for absolute essentials like GPU support. However, to achieve a certain level of throughput, crazy warmup time like several minutes is still unacceptable.
\end{enumerate}

\subsection{Security Solution}
There are three mainstream security solutions for our consideration: 1) virtual machine (VM) such as Virtualbox, 2) container such as Docker/Podman, and 3) sandbox such as Firejail. The primary areas of interest are compared in Table~\ref{tab:security-solutions}\footnote{There are some caveats to the claims listed in this table. For more details please refer to Appendix~\ref{as:comparison-of-security-solutions}.}:

\begin{table}[H]
\centering
\begin{tabular}{|c|c|cc|c|}
\hline
\multirow{2}{*}{} & \textbf{VM} & \multicolumn{2}{c|}{\textbf{Container}} & \textbf{Sandbox} \\ \cline{2-5} 
 & VirtualBox & \multicolumn{1}{c|}{Docker} & Podman & Firejail \\ \hline
\textbf{Rootless} & No & \multicolumn{1}{c|}{Yes} & Yes & Yes \\ \hline
\textbf{Level of isolation} & Very high & \multicolumn{2}{c|}{High} & Medium \\ \hline
\textbf{Overhead} & High & \multicolumn{2}{c|}{Low} & None \\ \hline
\textbf{Startup time} & $\sim$15s & \multicolumn{2}{c|}{$\sim$3s} & $\sim$0.05s \\ \hline
\textbf{GPU support} & No & \multicolumn{2}{c|}{Yes, with NVIDIA container runtime} & Yes \\ \hline
\end{tabular}
\caption{Comparison of Mainstream Security Solutions}
\label{tab:security-solutions}
\end{table}

There are several must-haves for the candidate solution:

\begin{enumerate}
    \item GPU-support. Many recent RL algorithms are practically impossible to run without a GPU. This eliminates VM without PCI passthrough.
    \item Server needs to be shared. This eliminates VM with PCI passthrough as it requires exclusive access. This also eliminates Podman and Docker, as they prevent the GPU from being shared by any other container with root access.
\end{enumerate}

Therefore, Firejail is the only option left. I built a custom security profile to expose only the necessary file system and devices to processes inside the sandbox. I also used Firejail to impose CPU affinity and RAM usage limit on each sandbox.

\subsection{Resource Awareness}
\label{ss:aivle-worker-resource-awareness}
The architecture of aiVLE Worker's resource awareness is illustrated in Figure~\ref{fig:aivle-worker-resource-awareness-arch}. In the following subsections, we will discuss the objectives and design of related components, namely \texttt{monitor} and \texttt{warden} modules.

\begin{figure}[H]
    \centering
    \includegraphics[width=0.8\textwidth]{images/aivle-worker-resource-awareness-arch.png}
    \caption{aiVLE Worker Resource Awareness Architecture}
    \label{fig:aivle-worker-resource-awareness-arch}
\end{figure}

\subsubsection{Resource Monitoring - \texttt{monitor} module}
\label{sss:monitor}
To achieve resource-sensitive load balancing as described in a later Section~\ref{ss:aivle-web-load-balancing}, and to enforce resource limit as described in Section~\ref{sss:warden}, we need to monitor CPU/GPU utilization and RAM/VRAM usage and take actions accordingly. Therefore we implement the \texttt{monitor} module inside aiVLE Worker that runs in parallel with the main worker process. The \texttt{monitor} module
\begin{enumerate}
    \item Monitors the system utilization periodically
    \item Controls the worker's task queue subscription according to prefetched threshold
    \item Sends monitoring metrics to the \texttt{warden} module to enable resource limit enforcement
\end{enumerate}

The first objective is achieved by the execution flow illustrated in Figure~\ref{fig:aivle-worker-monitor-flow} with the help of \href{https://pypi.org/project/psutil/}{\texttt{psutil}} for CPU/RAM monitoring and \href{https://github.com/fbcotter/py3nvml}{\texttt{py3nvml}} for GPU monitoring. The last objective involves inter-process communication that will be discussed later in detail with Figure~\ref{fig:aivle-worker-ipc}.

\begin{figure}[H]
    \centering
    \includegraphics[width=0.45\textwidth]{images/aivle-worker-monitor-flow.png}
    \caption{Execution Flow of \texttt{monitor} module}
    \label{fig:aivle-worker-monitor-flow}
\end{figure}

Now we explain why the \texttt{monitor} module is also responsible for controlling the task queue subscription: due to the limitation of Celery (the distributed task queue library we used to build the evaluation subsystem), the subscription control is unidirectional. In other words, Celery does not allow the worker itself to pause \textbf{consuming} tasks - the worker can only \textbf{shutdown} itself entirely, and only the master server can pause \textbf{sending} tasks to a certain worker. So we have two possible designs: either sending monitoring metrics periodically to the master server and let the master server have full control, or make the workers fetch the threshold on startup and request the master server to pause whenever the threshold is violated. The second option is our obvious choice because it incurs much less communication overhead and a fixed, prefetch threshold works fine in our use case.

\subsubsection{Limit Enforcement - \texttt{warden} module}
\label{sss:warden}
The \texttt{warden} module is responsible for:
\begin{enumerate}
    \item Collecting task information from \texttt{worker} module
    \item Collecting system resource information from \texttt{monitor} module
    \item Terminating sandboxes that violates resource restrictions and report to aiVLE Web\footnote{We use ``master server'' and aiVLE Web interchangeably in the context of task queue because they run in the same Python monolithic server application.}
\end{enumerate}

Terminating sandboxes and reporting to aiVLE Web is as straightforward as killing the corresponding PID and making an HTTP request. What makes \texttt{warden}'s job challenging is its first two objectives: there are three modules that run in parallel: \texttt{worker}, \texttt{monitor} and \texttt{warden}. And they need to exchange information in order to collaborate. This is where ZeroMQ~\cite{zeromq} comes handy\footnote{...again, see Section~\ref{ss:agent-env-communication}}: only the worker knows the mapping from sandbox PID to task information (i.e., VRAM limit), and only the \texttt{monitor} knows the VRAM usage of each process. Both need to send their data to the \texttt{warden} process which oversees all running jobs and terminates jobs that violates restrictions. Figure~\ref{fig:aivle-worker-ipc} shows the inter-process communication among these three:

\begin{figure}[H]
    \centering
    \includegraphics[width=0.8\textwidth]{images/aivle-worker-ipc.png}
    \caption{aiVLE Worker Inter-process Communication}
    \label{fig:aivle-worker-ipc}
\end{figure}

For \texttt{warden}'s case, every time it receives an update from \texttt{monitor}, it goes through the list of active sandboxes, queries the process hierarchy\footnote{This is necessary as most frameworks like PyTorch will spawn new processes for computation. Generally the process that is directly utilizing GPU is no longer the parent process. And there might be multiple processes in the same sandbox that consumes VRAM. By traversing the process tree recursively, we prevent intentional or unintentional attempts of ``downplaying'' the resource consumption of certain sandboxes.} to calculate the total VRAM usage of each sandbox, and finally checks if any sandbox needs to be terminated.

\section{aiVLE Web - AI competition platform}
\label{ch:aivle-web}
aiVLE Web consists of a Django backend that is $\sim$2.5K LoC and a React frontend that is $\sim$2K LoC. An active instance is deployed on the server and is accessible from \href{https://aivle.leotan.cn/}{https://aivle.leotan.cn/} (frontend) and \href{https://aivle-api.leotan.cn/}{https://aivle-api.leotan.cn/} (API and admin panel). Figure~\ref{fig:aivle-web-frontend-screenshot} shows a few screenshots of the React frontend. Do note that we used mobile-sized screenshots due to limited space, but the design is responsive (i.e., it adapts to different screen sizes).

\begin{figure}[H]
    \centering
    \includegraphics[width=\textwidth]{images/aivle-web-frontend-screenshot.png}
    \caption{aiVLE Web Frontend Screenshots}
    \label{fig:aivle-web-frontend-screenshot}
\end{figure}

\subsection{Design Goal}
There are three primary considerations to the design of aiVLE web: extensibility, scalability, and usability. 

\textbf{Extensibility} means the architecture should be flexible enough for future upgrades including but not limited to multi-agent task competition. Due to space limit we refrain from discussing the software engineering choices that are in favor of extensibility, but you may take a sneak peek at how to extend the existing system from Appendix~\ref{appendix:aivle-web_matchmaking}: \nameref{appendix:aivle-web_matchmaking}.

\textbf{Scalability} means the platform should be able to spread the evaluation workload, which is obviously the most computationally heavy task, over many worker machines. We will discuss this objective in detail with 1) highly available and fault tolerant task queue (Section~\ref{ch:aivle-web_highly-available-task-queue}) and 2) resource-sensitive load balancing (Section~\ref{ss:aivle-web-load-balancing}).

\textbf{Usability} means the platform should provide all the features the users (e.g., CS4246 teaching team) need for a successful semester of teaching. While there are many features we could discuss, also due to space limit, we pick the role-based permission model (Section~\ref{ss:aivle-web-permission-model}) as an typical example of how we carefully balance complexity and flexibility during the development process.

\subsection{Highly Available and Fault Tolerant Task Queue}
\label{ch:aivle-web_highly-available-task-queue}
This is a continuation of \hyperref[ss:aivle-worker-design-goal]{scalability of (grading) workers} (Section~\ref{ss:aivle-worker-design-goal}). In the section for aiVLE worker, we addressed the problem of running the worker on as many computers as possible with as little configuration/permission as possible. Here we address the problem of 1) coordinating the communication between the workers and the backend server, and 2) distributing grading tasks to the workers efficiently for shorter waiting time and higher resource utilization.
\subsubsection{The Problem}
In aiVLE 1.0, there was no concept of ``task queue''. All pending tasks are stored in the DB with the status ``QUEUED''. Communication between the worker (or runner as per the term used by original author) and server is \textit{half-duplex by polling}. In other words, the worker makes \textbf{periodic} requests to the server for new ungraded submissions:
\begin{figure}[H]
    \centering
    \includegraphics{images/aivle-web-old-task-queue.png}
    \caption{Old aiVLE “Task Queue”}
    \label{fig:aivle-web-old-task-queue}
\end{figure}
There are three critical limitations to this approach:
\begin{enumerate}
    \item Worker-side polling does not scale well: there is zero mechanism in orchestrating the timing and order of workers polling the server for new jobs. The severity of possible traffic spike (i.e., many workers poll the server at the same time due to lack of coordination) increases linearly with the number of worker nodes.
    \item Race condition: if two worker pull submissions at the same time, both will get the same ungraded submission. Such redundant work will get more significant with more worker nodes, which hurts the overall efficiency of the worker cluster.
    \item No concurrency on each worker\footnote{In the actual use of aiVLE 1.0, we run multiple worker clients on the same machine for disguised concurrency. Besides the inconvenience of manually starting multiple worker clients, this approach makes managing resources difficult as each client runs without knowing the existence of others even if they are on the same machine.\label{fn:worker-disguised-concurrency}}: each individual worker only polls for new job when it has no submission to grade. In other words, each worker can only grade one submission at a time.
    \item No load balancing: the backend has no control over which worker grades which submission, therefore load balancing is virtually impossible
\end{enumerate}

\subsubsection{The Solution}
Similar to the idea of extracting the responsibility of data storage/management into a separate database backend, we delegate the messaging tasks to a message queue (MQ). Conceptually, message queue enables asynchronous communication between clients (who submit tasks) and workers (who finish tasks). As for implementing the concept in our Python web application, we used \href{https://docs.celeryq.dev/en/stable/}{Celery} framework with \href{https://www.rabbitmq.com/}{RabbitMQ} message queue broker/backend. Figure~\ref{fig:aivle-web-mq} illustrates how the MQ-based task queue works:

\begin{figure}[H]
    \centering
    \includegraphics{images/aivle-web-mq.png}
    \caption{Message Queue Based Task Queue}
    \label{fig:aivle-web-mq}
\end{figure}

Every worker listens to one or more ``queues'', where the message broker is responsible for allocating tasks fairly among workers in each queue. When an evaluation job comes, aiVLE backend will submit the job to an appropriate queue (i.e., private queue if user has dedicated workers available, otherwise public CPU/GPU queue according to task specification) and wait for the assigned worker to submit evaluation result. The task will remain in the queue until it is processed (i.e., message queue is persistent). A randomly generated task ID is used to authenticate worker's submission – only the worker whom the broker assigned the task to will have this ID. This approach not only reduces the number of requests to be $O(n)$ where $n$ is the number of evaluation jobs, but also ensures fair distribution of tasks among workers. Moreover, we also benefit from other standard features of message queue such as automatic retry and heartbeat checks.

\subsection{Resource-sensitive Load Balancing}
\label{ss:aivle-web-load-balancing}
By using message queue for task distribution, we already have some primitive load balancing - Celery with RabbitMQ backend by default dispatches messages to all consumers in round-robin style, therefore all consumers are expected to consume the same number of tasks from the same queue over a fixed period of time. Although this is a huge improvement over no load balancing at all, from some stress testing using real-world workload, we find it necessary to take system load into consideration. Specifically, the primitive method of load balancing by number of tasks works poorly when certain tasks are much more resource intensive. For example, assume both worker A and worker B receive 5 submissions, but the ones for worker A consumes 30\% of total VRAM while the ones for worker B takes only 10\% of total VRAM. There are two serious implications in this imaginary scenario:
\begin{enumerate}
    \item \textbf{Unfairness}: suppose the worker is configured to run at most 8 concurrent evaluations, the fourth and fifth submissions arriving at worker A will not have sufficient VRAM. They will likely receive runtime errors which are entirely ours to blame.
    \item \textbf{Inefficiency}: suppose the task queue is configured to automatically retry the failed job by putting a new evaluation job into the same queue, since worker A is still accepting submissions, the retry attempt may take additional fails to finally arrive at worker B.
\end{enumerate}

The key to solving such unwanted behavior is 1) to monitor the available system resources, 2) to stop processing new tasks when the available resources fall below certain thresholds. Most of the heavy-lifting is handled on the worker side (see Section~\ref{ss:aivle-worker-resource-awareness}). As for aiVLE Web, it merely uses \texttt{add\_consumer} and \texttt{cancel\_consumer} Celery APIs for resuming and pausing consuming respectively. Nevertheless, there are two points to note in this solution:

\begin{enumerate}
    \item It does not cancel already running jobs. Instead, it only stops dispatching new tasks to the worker where the threshold is met. This behavior is acceptable in our case since we only need to promise students with a certain amount of resources to run their submissions. And our solution guarantees\footnote{Technically speaking, this guarantee does not hold between two checks on system information, but we can easily reduce the impact by performing the check more frequently. In our experience, one second interval is more than enough.} the promised amount of CPU/RAM/VRAM at the beginning of any evaluation job.
    \item It does not balance the resource utilization among all workers. \texttt{cancel\_consumer} simply removes the worker from the list of consumers to dispatch tasks to. To achieve resource utilization balance, aiVLE Web needs to understand each worker's utilization in real-time and dispatch tasks accordingly. We think the overhead (i.e., network traffic for reporting utilization data) and complexity (i.e., load balancing algorithm) of this design significantly outweighs the potential benefits.
\end{enumerate}

Besides addressing the previously mentioned two serious implications, surprisingly, resource-sensitive load balancing greatly increases the system utilization. As mentioned in Footnote~\ref{fn:worker-disguised-concurrency}, without resource awareness, aiVLE 1.0 could only achieve disguised concurrency by running multiple worker clients on the same machine. To ensure the machines will not be overloaded, we could only ``play it safe'' by always under-utilizing the resources. In fact, the current configuration of aiVLE 1.0 only allows 3 concurrent evaluations, resulting in an average of $\sim30\%$ of volatile GPU usage. In comparison, as we will show later in Section~\ref{ss:load-balancing-exp-results}, we can achieve up to $>90\%$ GPU utilization rate, and with the later introduced resource-sensitive load balancing, overloading the system is no longer a concern. This translates to roughly three times\footnote{In real world use you might not want to set the threshold to be as high as 90\%, but even with a conservative 80\% threshold the improvement is still very significant.} of a performance improvement.

\subsection{Role-based Permission Model}
\label{ss:aivle-web-permission-model}
In Django we have two major types of permission model: per-model permission (built-in authentication system) and per-object permission (\href{https://github.com/django-guardian/django-guardian}{django-guardian}). Per-model permission means the finest granularity is model-level, that is we may only check if the user has read/write/edit/delete access to the model. Per-object permission is the finest granularity possible as we can establish arbitrary permission between any object and any user. In fact, the worst possible space complexity is $O(mnk)$ where $m$ is the number of objects, $n$ is the number of users and $k$ is the number of permissions.

In our use case, model-based permission is too coarse-grained: a student should only be able to view the tasks in the course he/she enrolled in, rather than all the tasks available on the platform. On the other hand, object-based permission is too overkill: there might be hundreds of thousands of job records in the database, storing each user's permission to each job record is simply too wasteful\footnote{This can be avoided by properly configuring django-guardian, but its flexibility does allow arbitrary permission on any object as mentioned earlier. In fact, if we restrict django-guardian to avoid such wasteful behavior, we essentially have the soon-to-be-discussed role-based permission model.}. Therefore, we propose a solution that is the sweet spot between the two extremes: role-based permission model.

The entity relationship diagram of aiVLE Web is illustrated in Figure~\ref{fig:aivle-web-er-diagram}:
\begin{figure}[H]
    \centering
    \includegraphics[width=0.8\textwidth]{images/aivle-web-er-diagram.png}
    \caption{aiVLE Web Entity Relationship Diagram}
    \label{fig:aivle-web-er-diagram}
\end{figure}

In the relational entity Participation, we record the ``role'' of the User in a Course. Possible roles\footnote{Note that a user can have different roles in different courses. And superuser who has access to absolutely everything is out of discussion here.} (as defined in \texttt{app.model.participation}) are: administrator (ADM), lecturer (LEC), teaching assistant (TA), student (STU) and guest (GUE).

The idea of role-based permission model is centered around one's role in the corresponding course: to determine one's access to any object, we first find the object's related course, then check if the user has access to this object in the context of this related course. For example, if we need to know if the user has view access to a certain \texttt{Job}, we can follow the arrows in the diagram: Job $\to$ Submission $\to$ Task $\to$ Course $\to$ Participation $\to$ User to find the corresponding role\footnote{The utility function \texttt{has\_perm} automatically queries the role given the course and the user, so its argument is the course instead of the role.}, and check if this role has \texttt{job.view} permission according to the permission lookup table (see Table~\ref{tab:aivle-web-permission-table}).

By introducing the \texttt{Participation} relational entity and \texttt{has\_perm(course, user, permission)} utility function, we compressed the worst case $O(mnk)$ space complexity to $O(nk)$ and it is a significant improvement. In reality the number of objects $m$ significantly outweighs the number of users $n$ or permissions $k$. We can summarize different roles' access (as defined in aiVLE.settings.ROLES) using a permission matrix in Table~\ref{tab:aivle-web-permission-table}:

\begin{table}[H]
\centering
\begin{tabular}{|c|l|l|l|l|l|l|}
\hline
\multicolumn{1}{|l|}{} &  & Admin & Lecturer & TA & Student & Guest \\ \hline
\multirow{5}{*}{Task} & View opened tasks & x & x & x & x & x \\ \cline{2-7} 
 & View all tasks & x & x & x &  &  \\ \cline{2-7} 
 & Add task & x & x &  &  &  \\ \cline{2-7} 
 & Edit task & x & x & x &  &  \\ \cline{2-7} 
 & Delete task & x & x &  &  &  \\ \hline
\multirow{4}{*}{Submission} & View own submissions & x & x & x & x &  \\ \cline{2-7} 
 & View all submissions & x & x & x &  &  \\ \cline{2-7} 
 & Add submission (under own name) & x & x & x & x &  \\ \cline{2-7} 
 & Download submission & x & x & x &  &  \\ \hline
\end{tabular}
\caption{Partial Permission Lookup Table}
\label{tab:aivle-web-permission-table}
\end{table}
\chapter{Deployment and Testing}
\label{ch:deployment-and-testing}
As mentioned in Section~\ref{ch:design-and-impl}, aiVLE 2.0, especially aiVLE Worker and aiVLE Grader, is designed to be highly scalable and easy to deploy. However, just like any distributed systems, being able to run the individual components on a single machine is one thing, having all the components cooperate on separate machines to achieve actual distribution is another. Thus, to pave way for future production deployment, and to demonstrate the actual performance of such a distributed system, we performed a complete deployment using several SoC cluster nodes\footnote{Special thanks to the SoC compute cluster admins and Dr. Narayan for their help in securing these very powerful GPU nodes as shown later in Section~\ref{ss:deployment-environment}.} and did several experiments/benchmarks on the system.

There will be three sections covering both the deployment and tests this chapter: Section~\ref{s:deployment} describes the deployment environment, deployment steps and issues we encountered before making the system fully functional on SoC. Section~\ref{s:load-balance-exp} describes the methodology and findings of load balancing among \textbf{multiple} worker nodes. Section~\ref{s:concurrency-exp} describes the methodology and findings of running evaluation jobs concurrently on a \textbf{single} worker node.

\section{Deployment}
\label{s:deployment}
Note that this deployment happened during the winter break (Nov 2021 to Jan 2022). Therefore some later updates to the aiVLE Web and Worker are not included during this deployment (most notably, the resource-sensitive load balancing related features). However, this should not affect the later mentioned test results as the experiments are designed to have no system overloading (see Section \ref{sss:choice-of-params}).

In specific, the exact versions used are (Git commit hash with GitHub link):
\begin{itemize}
    \item aiVLE Web: \href{https://github.com/edu-ai/aivle-web/commit/b346a68e30aa05656ae2e6cc106414bcba5430fd}{b346a68e30aa05656ae2e6cc106414bcba5430fd}
    \item aiVLE Worker: \href{https://github.com/edu-ai/aivle-worker/commit/647d767a72acf6c5f5ded533d77141ca91f4ef9d}{647d767a72acf6c5f5ded533d77141ca91f4ef9d}
\end{itemize}

\subsection{Environment}
\label{ss:deployment-environment}
aiVLE Worker is deployed on SoC compute cluster \texttt{xgpg0, xgpg1, xgpg2} with the following configuration:
\begin{itemize}
    \item Operating System (Code~\ref{code:deployment-os}): Ubuntu 20.04 LTS with Linux kernel version 5.4.0
    \item GPU (Code~\ref{code:deployment-gpu}): NVIDIA A100-PCI, Driver 495.29, CUDA 11.5
    \item CPU: 2x AMD Epyc 7352, in total 48 cores/96 threads, base clock 2.3GHz
    \item RAM: 256GiB DDR4
\end{itemize}

\begin{code}
\begin{minted}[frame=lines,framesep=2mm,baselinestretch=1.2,bgcolor=LightGray,fontsize=\footnotesize,linenos]{shell}
> cat /proc/version
Linux version 5.4.0-91-generic (buildd@lcy01-amd64-017) 
(gcc version 9.3.0 (Ubuntu 9.3.0-17ubuntu1~20.04)) #102-Ubuntu SMP Fri Nov 5 16:31:28 UTC 2021
\end{minted}
\captionof{listing}{Deployment Environment - Operating System}
\label{code:deployment-os}
\end{code}

\begin{code}
\begin{minted}[frame=lines,framesep=2mm,baselinestretch=1.2,bgcolor=LightGray,fontsize=\footnotesize,linenos]{shell}
> nvidia-smi 
Sat Jan  1 13:32:49 2022       
+-----------------------------------------------------------------------------+
| NVIDIA-SMI 495.29.05    Driver Version: 495.29.05    CUDA Version: 11.5     |
|-------------------------------+----------------------+----------------------+
| GPU  Name        Persistence-M| Bus-Id        Disp.A | Volatile Uncorr. ECC |
| Fan  Temp  Perf  Pwr:Usage/Cap|         Memory-Usage | GPU-Util  Compute M. |
|                               |                      |               MIG M. |
|===============================+======================+======================|
|   0  NVIDIA A100-PCI...  On   | 00000000:01:00.0 Off |                    0 |
| N/A   49C    P0    36W / 250W |      0MiB / 40536MiB |      0%      Default |
|                               |                      |             Disabled |
+-------------------------------+----------------------+----------------------+
\end{minted}
\captionof{listing}{Deployment Environment - GPU}
\label{code:deployment-gpu}
\end{code}

aiVLE Web is deployed on a \href{https://linode.com/}{Linode} Nanode 1GB VPS (virtual private server) with 1 virtual CPU core and 1 GiB of RAM.

\subsection{Limitations}
\label{ss:deployment-limitations}
There is one additional problem with distributed systems that I did not mention in the introductory paragraph of this chapter: even if the system works on a few machines, it does not necessarily mean the system will work or scale well to dozens or even hundreds of machines. While we design the system to be highly scalable, and the underlying technologies have proven to be effective on hundreds of machines, we are never certain until we actually scale the system to that many nodes and put it under pressure.

Unfortunately, for the time being, we are unable to materialize such a large-scale experiment: SoC compute cluster does not have that many GPU nodes available, nor could I find a demanding enough task to put the system at such a scale under considerable pressure. In fact, on the \texttt{xgpg*} nodes used for this experiment, every evaluation takes $\sim$10 seconds to finish, which is too short for even tens of worker nodes: for the evaluation subsystem to be stressed, we at least need to keep the task queue ``filled''. In other words, the rate of submitting new jobs into the task queue should be much greater than the rate of workers finishing jobs. However, with each job only taking $\sim$10 seconds, suppose we have 50 workers, our rate of consumption would be $50/10=5$ jobs per second (every 10 seconds we can process 50 jobs). If we assume each submission to be $\sim$20MiB\footnote{Many deep learning models are much larger than this, so we are giving a conservative estimation here}, it would require at least 100MiB per second of network speed, which is already approaching the limit of Gigabit Ethernet available on most machines.

Therefore, with our current resources, it is nearly impossible to properly test the scalability potential of aiVLE 2.0. However, this does not mean that our experiments are useless! Quite the contrary, our experiment setup would be very similar to actual resources available for CS4246 teaching team, and we expect aiVLE 2.0 have a much higher utilization rate of computational power - as a result, process submissions with smaller delays.

In short, we remain cautiously optimistic about the scalability of aiVLE 2.0 as we do not yet have experiment data to support it, but we are confident about its capability on supporting CS4246 from the experiments.

\subsection{Steps}
To simulate the production deployment process, we scrapped the existing setups and started everything afresh. The following steps are sufficient for any deployment from the ground up:

\begin{enumerate}
    \item Prepare the message queue broker: either by installing \href{https://www.rabbitmq.com/}{RabbitMQ} or using cloud message queue provider such as \href{https://www.cloudamqp.com/}{CloudAMQP}. In our case, we installed RabbitMQ on the same VPS with aiVLE Web.
    \item Install and start aiVLE Web on the VPS.  \href{https://github.com/edu-ai/aivle-web#readme}{aiVLE Web Readme} describes this process in detail.
    \item Setup the users, courses, tasks in aiVLE Web via its RESTful API or Django admin panel.
    \item Prepare the worker nodes with necessary dependencies (i.e., Firejail, Pip, Virtualenv, CUDA drivers). In our case, we requested the cluster admins to install Firejail as all other dependencies are already available.
    \item Install and start aiVLE Worker on the worker nodes. \href{https://github.com/edu-ai/aivle-worker#readme}{aiVLE Worker Readme} describes this process in detail.
\end{enumerate}

\subsection{Issues and Solutions}
There are some issues we found during the deployment process. None of which is critical in a sense that we eventually found workarounds without modifying any existing codebase or design. But we think it is nonetheless useful to point them out here as potential users of this system is likely to encounter some of them as well.

\subsubsection{Switching broker from RabbitMQ to AWS SQS}
If the worker node resides behind a firewall that restricts access to certain ports (especially under a whitelist policy where only selected ports are available), then you are likely to find RabbitMQ unusable - its underlying protocol, Advanced Message Queuing Protocol (AMQP), uses port 5671/5672 by default. There are three possible solutions:
\begin{enumerate}
    \item Change RabbitMQ port to one that is allowed by the worker node firewall. Do note that AMQP is not based on HTTP/HTTPS, so when change the port to 80/443 you need to ensure that not only the broker server has those ports available, but also the worker node firewall DOES allow non-HTTP traffic via port 80/443.
    \item Deploy RabbitMQ internally. If the firewall only blocks external access and allows internal access to all ports (like SoC firewall), and all worker nodes reside within the local network, then deploying the RabbitMQ inside the local network would be an uncompromising\footnote{Given you have enough privilege to install RabbitMQ on one of the machines, of course. It also requires root access, which I did not have.} solution.
    \item Switch RabbitMQ to HTTP/HTTPS-based broker such as \href{https://aws.amazon.com/sqs/}{AWS SQS}. Do note that this solution greatly affects the capability of Celery task scheduling - for example, remote worker control is impossible with SQS. As a result, later advanced features like resource-sensitive load balancing would not work under SQS.
\end{enumerate}
In our case, the SoC firewall blocks most ports on external IP address, and forcing RabbitMQ to use port 80 was futile. In the end we switched to AWS SQS as a workaround\footnote{It did not affect any existing functionality then - features such as resource-sensitive load balancing came later in the second semester.}.

\subsubsection{Firejail Version Requirement}
Firejail security profile is not forward compatible, and the latest Firejail version is determined by the OS version, therefore the default security profile of aiVLE Worker doesn't work on  \texttt{xgpd0} which has Ubuntu 16.04 installed. It works as expected on both Ubuntu 18.04 and Ubuntu 20.04 with their latest Firejail version respectively.

To avoid future confusions, we listed Ubuntu 20.04 as a requirement for aiVLE Worker on its Readme file.

\subsubsection{PyTorch Installation Issue with Latest nVIDIA GPU}
Although it has been two years since the launch of RTX 30-series GPU\footnote{The same applies to data center GPUs launched after RTX 30-series, such as nVIDIA A100 used in our deployment.}, PyTorch official PIP channel still hasn't supported CUDA 11 (which is the minimum CUDA version for 30-series GPU). So instead of 
\begin{code}
\begin{minted}[frame=lines,framesep=2mm,baselinestretch=1.2,bgcolor=LightGray,fontsize=\footnotesize]{shell}
pip3 install torch torchvision torchaudio
\end{minted}
\end{code}
we need to use
\begin{code}
\begin{minted}[frame=lines,framesep=2mm,baselinestretch=1.2,bgcolor=LightGray,fontsize=\footnotesize]{shell}
pip3 install torch==1.10.1+cu113 torchvision==0.11.2+cu113 \
torchaudio==0.10.1+cu113 -f \
https://download.pytorch.org/whl/cu113/torch_stable.html
\end{minted}
\end{code}

This means the author of the task needs to be aware of the CUDA version of their allocated grading machines, and adjust their \texttt{requirements.txt} accordingly.

\section{Load Balance Experiment}
\label{s:load-balance-exp}
Raw logs and analyzing scripts can be found in \href{https://github.com/edu-ai/aivle-experiment-logs}{aiVLE Experiment Logs}. For correspondence between experiment setup and log file index, please refer to Table~\ref{tab:load-balance-exp}

\begin{table}[H]
\centering
\begin{tabular}{|c|c|c|c|c|c|}
\hline
\multicolumn{1}{|l|}{\begin{tabular}[c]{@{}l@{}}Web\\ log index\end{tabular}} & \multicolumn{1}{l|}{\begin{tabular}[c]{@{}l@{}}Worker\\ log index\end{tabular}} & \multicolumn{1}{l|}{Node count} & \multicolumn{1}{l|}{\begin{tabular}[c]{@{}l@{}}Concurrency \\ of worker\end{tabular}} & \multicolumn{1}{l|}{Submission count} & \multicolumn{1}{l|}{\begin{tabular}[c]{@{}l@{}}Concurrency \\ of submission\end{tabular}} \\ \hline
5 & 2 & 3 & 8 & 100 & 100 \\ \hline
6 & 3 & 2 & 8 & 100 & 100 \\ \hline
7 & 4 & 1 & 8 & 100 & 100 \\ \hline
\end{tabular}
\caption{Load Balancing Experiment Setup}
\label{tab:load-balance-exp}
\end{table}

\subsection{Methodology}
\label{ss:lb-exp-meth}
Since the evaluation task and all worker nodes are exactly the same, we have four parameters/variables to adjust:
\begin{enumerate}
    \item Node count: number of active nodes during the experiment.
    \item Concurrency of worker: maximum number of concurrent evaluation jobs allowed on \emph{each} worker node.
    \item Submission count: number of \emph{total} evaluation jobs submitted to the task queue.
    \item Concurrency of submission: maximum number of threads submitting jobs at the same time.
\end{enumerate}

In this section and Section~\ref{s:concurrency-exp}, submission count and concurrency of submission are both 100. This means before the first job is assigned to any of the worker nodes, 100 jobs are already queued. This is to ensure there will always be sufficient tasks for the workers to work on - if concurrency of submission is significantly smaller than the number of submissions, then the rate of submitting job may dictate the rate of workers finishing jobs, which is undesirable for our stress-oriented experiments.

For load balance experiment, we \textbf{fix concurrency on all workers to be 8}, measure \textbf{time taken} to finish 100 submissions with
\begin{itemize}
    \item 1 node (\texttt{xgpg0})
    \item 2 nodes (\texttt{xgpg0,1})
    \item 3 nodes (\texttt{xgpg0,1,2})
\end{itemize}

On the master server (aiVLE Web), we logged the critical phases of each submission with a timestamp, in specific, there is a timestamped record when
\begin{enumerate}
    \item received submission
    \item submitted to task queue
    \item task picked up by a worker
    \item task being worked on by a worker
    \item task terminated (finished or failed)
\end{enumerate}

The start time is defined to be the earlier of 1) the latest "received submission" record and 2) the earliest "task picked up by a worker" record. The finish time is defined to be the latest "task terminated" record. The total time is defined to be the difference between the finish time and the start time. The detailed method of calculating time taken to finish all submissions can be found in the \href{https://github.com/edu-ai/aivle-experiment-logs/blob/main/web/analyze.ipynb}{analyze script}.

On each worker node, we also logged the timestamped GPU utilization rate and VRAM usage periodically. This helps us understand whether all worker nodes are busy most of the time - unnecessary idling is a sign of poor load balancing.

\subsubsection{Choice of Parameters}
\label{sss:choice-of-params}

There are two parameters of our choice: number of total submissions (100) and maximum concurrent evaluations on each worker (8). These are not randomly picked nor empirical choices.

For the maximum concurrency allowed, we want the concurrency to be as large as possible without overwhelming the system. Since our evaluation task is mostly GPU-bound, we performed experiments on a \textbf{single} machine with concurrent evaluation ranging from 2 to 16 and observed both the GPU utilization and VRAM utilization. With 40GiB of VRAM, we cannot overload the VRAM with even 16 concurrent jobs as each job takes less than 2GiB of VRAM. On the other hand, GPU utilization increases linearly with number of concurrent jobs until more than 8 jobs. Increasing more jobs will not significantly increase the GPU utilization, indicating that the machine could only handle up to 8 jobs without significantly degraded per-job performance.

For the total submissions to finish, more submissions means the average number is more representative, but too many means much longer waiting time. We performed experiments on a \textbf{single} machine with 1000, 500 and 100 submissions. We find the time taken to be almost perfectly linear, while 1000 submissions can take more than 30 minutes to finish. Therefore 100 submissions is considered to be a sweet spot between accuracy and convenience.

\subsection{Results}
\label{ss:load-balancing-exp-results}

For the performance of load balancing, below are the times for each test case:
\begin{itemize}
    \item 1 node: 235.426s (baseline)
    \item 2 nodes: 128.261s (91.78\%)
    \item 3 nodes: 92.475s (84.86\%)
\end{itemize}
The percentage is the scaling efficiency defined by $\frac{\text{Optional time}}{\text{Measured time}}$ where \emph{Optimal Time} is defined as (take $N$ as the number of nodes, $t_0$ as the time taken with one node) $\frac{t_0}{N} \times 100\%$.

For the fairness of load balancing, below are the number of jobs assigned to each worker node (when there are more than one node).
\begin{itemize}
    \item 2 nodes: \texttt{\{'celery@xgpg0': 50, 'celery@xgpg1': 50\}}
    \item 3 nodes: \texttt{\{'celery@xgpg0': 34, 'celery@xgpg1': 36, 'celery@xgpg2': 30\}}
\end{itemize}

For the utilization of system resources, since our task is GPU-bound, below is the GPU and VRAM utilization plot of one of the worker nodes (others are similar):

\begin{figure}[H]
    \centering
    \includegraphics[width=0.8\textwidth]{images/experiment-lb-utilization-plot.png}
    \caption{GPU/VRAM Utilization Plot from \texttt{xgpg0\_2.log}}
    \label{fig:experiment-lb-utilization-plot}
\end{figure}

\section{Concurrency Experiment}
\label{s:concurrency-exp}
Raw logs and analyzing scripts can be found in \href{https://github.com/edu-ai/aivle-experiment-logs}{aiVLE Experiment Logs Repo}.For correspondence between experiment setup and log file index, please refer to Table~\ref{tab:concurrency-exp}

\begin{table}[H]
\centering
\begin{tabular}{|c|c|c|c|c|c|}
\hline
\multicolumn{1}{|l|}{\begin{tabular}[c]{@{}l@{}}Web\\ log index\end{tabular}} & \multicolumn{1}{l|}{\begin{tabular}[c]{@{}l@{}}Worker\\ log index\end{tabular}} & \multicolumn{1}{l|}{Node count} & \multicolumn{1}{l|}{\begin{tabular}[c]{@{}l@{}}Concurrency \\ of worker\end{tabular}} & \multicolumn{1}{l|}{Submission count} & \multicolumn{1}{l|}{\begin{tabular}[c]{@{}l@{}}Concurrency \\ of submission\end{tabular}} \\ \hline
7 & 4 & 1 & 8 & 100 & 100 \\ \hline
8 & 5 & 1 & 4 & 100 & 100 \\ \hline
9 & 6 & 1 & 2 & 100 & 100 \\ \hline
10 & 7 & 1 & 1 & 100 & 100 \\ \hline
\end{tabular}
\caption{Per-worker Concurrency Experiment Setup}
\label{tab:concurrency-exp}
\end{table}

\subsection{Methodology}
For explanation of parameters/variables, and the method of measuring total time please refer to Section~\ref{ss:lb-exp-meth}. For the per-worker concurrency experiment, we activate \textbf{only one} worker node, and measure \textbf{time taken} to finish 100 submissions with concurrency of worker ranging from 1, 2, 4 and 8.

\subsection{Results}

\begin{itemize}
    \item concurrency = 1: 1558.811s (baseline)
    \item concurrency = 2: 859.237s (90.71\%)
    \item concurrency = 4: 422.744s (92.18\%)
    \item concurrency = 8: 235.426s (82.77\%)
\end{itemize}

Similar to section \ref{ss:load-balancing-exp-results}, the percentage in the end is the scaling efficiency. The definition is also similar by changing the meaning of $N$ to the concurrency number.
\chapter{Conclusion}
\label{ch:conclusion}

\section{Summary}
\label{s:conclusion-summary}
We started this project in pursuit of a complete solution for RL algorithm evaluation that is extensible, scalable, and easy-to-use. We think these three objectives are critical for the success of such a platform, the lack of which on existing solutions motivated us to build one that is up to our standard.
During the two semester of this final year project, we
\begin{itemize}
    \item Implemented RL environment framework that supports agent-environment separation and multi-agent competition.
    \item Provided automated and secure evaluation for RL tasks.
    \item Design and deployed distributed evaluation subsystem based on message queue.
    \item Improved the web application for hosting competitions by 1) restructuring code, 2) comprehensive documentation, 3) adding new features (e.g., email verification, password reset, invitation token, course whitelist, etc.).
\end{itemize}

For the RL environment framework, we have invited another FYP student Ho Hol Yin for an early adoption in his project ``A unified testbed for AI teaching and research''. The feedback has been positive especially in terms of OpenAI Gym compatibility.

For the evaluation subsystem, we deployed the system to SoC compute cluster during the winter break and performed several performance experiments. We have validated the fair distribution of evaluation jobs, achieved $\sim$3 times of utilization improvement over the previous solution and demonstrated the new system’s scaling capability of supporting CS4246.

For the competition-hosting web platform, by working with Dr. Narayan closely, we have addressed most of the pain points from CS4246 teaching team. More importantly, the documentation is now sufficient for fresh deployments and future upgrades/maintenance.

\section{Limitations}
\label{s:conclusion-limitations}
While we think this project has been going generally smoothly, and the outcome is satisfactory, we must admit the limitations of the work done:

First, as mentioned in Section~\ref{ss:deployment-limitations}, due to limited resources, we could not test aiVLE 2.0's scalability on more machines (say dozens or even hundreds of nodes). Once we put aiVLE 2.0 into CS4246 production environment, we wish to onboard more stakeholders so that we can gradually scale up the system and explore its true potentials.

Second, aiVLE 2.0 does not provide a production-ready multi-agent competition solution. We started strong with aiVLE Gym that works great for multi-agent competitions, but once we try to extend aiVLE Web to support multi-agent tasks, we encountered much more difficulties than we anticipated. Most notably, we found it challenging to implement a skill rating and matchmaking\footnote{Here ``match’’ is equivalent to competition. ``Matchmaking'' means making matches that having the outcome of which significantly reduces our uncertainty of the competitors' skill ratings. Having similar skill ratings is only one aspect of matchmaking: the system gains little new knowledge by having the same pair compete against each other repeatedly, even if they have very similar skill ratings.} system that balances efficiency and fairness. Nevertheless, we managed to implement the infrastructure and abstractions for the matchmaking system (see Appendix~\ref{appendix:aivle-web_matchmaking}), and we definitely hope to materalize the proposal by implementing actual matchmaking algorithms on the infrastructure.

Lastly, although I emphasized the importance of maintainability repeatedly in this project by adopting many software engineering best practices such as comprehensive documentation and maintainable architecture, there is still some room for improvement, especially on the frontend code – unlike backend development that I am familiar with, this project is my first serious attempt with frontend development. Looking at the codebase after a year of experience, we think the frontend code needs considerable reorganization to achieve similar level of maintainability as other codebases.

\section{Future Work}
\label{s:conclusion-future_work}
Instead of calling aiVLE 2.0 a finished project, we think it is more appropriate to call it a foundation\footnote{...that is built upon the foundation of aiVLE 1.0} of a truly comprehensive AI competition platform. Our efforts in improving the project's maintainability are not only for the operation of CS4246 assignment grading, but also for the future upgrades from whoever interested in making it even more advanced.

In particular, as mentioned in Section~\ref{s:conclusion-limitations}, a concrete implementation of multi-agent matchmaking and skill rating system would be a huge upgrade - it opens up the possibility of multi-agent tasks that could be more interactive and competitive.

For the evaluation subsystem, we feel there is still room for improvement w.r.t. load balancing strategy - currently we distribute tasks fairly by round-robin, and pauses assigning new jobs to nodes that are under pressure. However, if we managed to solve the significant communication overhead of reporting utilization data to the master server in real-time, we could implement some load balancing algorithm that balances the system load among the workers.

And of course, as mentioned in Chapter~\ref{ch:design-and-impl}, aiVLE Gym is designed to function independently outside of aiVLE platform. Its capability of separating environment from agents could be useful for academic research in, say, human-in-the-loop machine learning.


\bibliographystyle{socreport}
\bibliography{socreport}

\appendix
\chapter{List of Links}
\label{appendix:links}

\section{Deployed Website}
\begin{itemize}
    \item Main website: \href{https://aivle.leotan.cn/}{aivle.leotan.cn}
    \item Administration site: \href{https://aivle-api.leotan.cn/api/v1/}{aivle-api.leotan.cn}
    \item Web API explorer: \href{https://aivle-api.leotan.cn/swagger/}{aivle-api.leotan.cn/swagger}
\end{itemize}

\section{GitHub Repositories}
\label{as:links-source_code}
\begin{itemize}
    \item Source code
    \begin{itemize}
        \item \href{https://github.com/edu-ai/aivle-gym}{aiVLE Gym}
        \item \href{https://github.com/edu-ai/aivle-grader}{aiVLE Grader}
        \item \href{https://github.com/edu-ai/aivle-worker}{aiVLE Worker}
        \item \href{https://github.com/edu-ai/aivle-web}{aiVLE Web Backend}
        \item \href{https://github.com/le0tan/aivle-fe}{aiVLE Web Frontend}
        \item \href{https://github.com/edu-ai/aivle-cli}{aiVLE CLI}
    \end{itemize}
    \item Experiment data: \href{https://github.com/edu-ai/aivle-experiment-logs}{Experiment Logs}
\end{itemize}

\section{PyPI Packages}
\begin{itemize}
    \item \href{https://test.pypi.org/project/aivle-gym/}{aiVLE Gym}
    \item \href{https://test.pypi.org/project/aivle-grader/}{aiVLE Grader}
    \item \href{https://test.pypi.org/project/aivle-worker/}{aiVLE Worker}
\end{itemize}

\section{Documentation}
\label{as:links-documentation}
\begin{itemize}
    \item Official documentation website: \href{https://edu-ai.github.io/aivle-docs/}{aiVLE Docs}
    \item Engineering documentation (i.e., inner-workings and design considerations)
    \begin{itemize}
        \item \href{https://yuanhong.larksuite.com/docs/docusSYdnLXZBojin39b8DGzKMT}{aiVLE Gym} (Password: mLpj)
        \item \href{https://yuanhong.larksuite.com/docs/docuseeHRJWAMV3p3uL7yYCOeYx}{aiVLE Grader} (Password: ZELI)
        \item \href{https://yuanhong.larksuite.com/docs/docussD8ik4yBXShA5kPyRGhgdg}{aiVLE Worker} (Password: qgF2)
        \item \href{https://yuanhong.larksuite.com/docs/docusfWZk1oYG8qkEMG7y2oxkye}{aiVLE Web} (Password: Z3Wn)
    \end{itemize}
\end{itemize}
\chapter{aiVLE Gym Design}
\label{appendix:aivle-gym}
\section{Multi-agent Communication DFA}
\label{as:aivle-gym_dfa}

In this section we define the following DFA more rigorously 

In this section we will define the aiVLE Gym communication DFA in a more mathematical (i.e., ``rigorous'' and detailed) fashion:

\begin{figure}[H]
    \centering
    \includegraphics{images/aivle-gym-multi-dfa.png}
    \caption{aiVLE Gym Multi-agent Communication DFA}
    \label{fig:aivle-gym_dfa}
\end{figure}

\begin{enumerate}
    \item States: INITIAL, WAIT\_RESET, WAIT\_ACTION, STEP, CLOSE
    \item Initial state $q_0$: INITIAL
    \item Accept (terminal) states $F$: CLOSE
    \item Input symbols: method (e.g. reset/step/close) and other conditions
\end{enumerate}

To make this DFA a mathematically rigorous one, the domain of transition function needs to be the Cartesian product of input symbols and states. However, since there are many symbols and states involved, listing them exhaustively takes too much space. For the sake of simplicity, we omitted many self-transitioning paths - if transitioning condition is not satisfied, we assume there's a self-transition path. ``Meaningful'' transitions are described below:

\begin{itemize}
    \item T1
    \begin{itemize}
        \item condition: method is "reset"
        \item artifact: reset the underlying base Gym environment, label this agent as already reset, save this agent's router ID
    \end{itemize}
    \item E1
    \begin{itemize}
        \item condition: method is "reset" and this sender hasn't reset before
        \item artifact: label this agent as already reset, save this agent's router ID, trigger an input symbol of "E1" (This trigger is conceptually equivalent immediately checking if we can transit to the next state. Details can refer to the implementation.)
    \end{itemize}
    \item T2
    \begin{itemize}
        \item condition: input symbol of "E1", all agents are labelled as have reset
        \item artifact: send initial observation to all agents, clear reset labels, clear router ID mappings
    \end{itemize}
    \item T3
    \begin{itemize}
        \item condition: method is "step"
        \item artifact: label this agent as already stepped, save this agent's router ID, save this agent's action, trigger an input symbol of "T3"
    \end{itemize}
    \item T4
    \begin{itemize}
        \item condition: input symbol of "T3", all agents are labelled as have stepped
        \item artifact: step forward in the base environment, send observation/reward/done/info to all agents, trigger an input symbol of "T4"
    \end{itemize}
    \item T5
    \begin{itemize}
        \item condition: input symbol is "T4", some of the agents still have ongoing episode
        \item artifact: clear "have stepped" labels on all agents, clear router ID mappings
    \end{itemize}
    \item T6
    \begin{itemize}
        \item condition: input symbol is "T4", none of the agents still have ongoing episode
        \item artifact: same as T5
    \end{itemize}
    \item T7
    \begin{itemize}
        \item condition: method is "close"
        \item artifact: close the underlying environment
    \end{itemize}
\end{itemize}

By implementing this DFA carefully, the multi-agent judge environment abstract base class is capable of handling any order of incoming agent requests. Most importantly, agent-side can expect responses synchronously therefore keep all their expectations about a normal single-agent Gym environment. Note that these intricate details are not of users' (both the agent author and environment author) concern. The environment author only needs to provide implementations for the abstract methods and our library will handle the rest.
\chapter{aiVLE Worker Design}
\label{appendix:aivle-worker}

\section{Comparison of Mainstream Security Solutions}
\label{as:comparison-of-security-solutions}

There are several ``asterisks'' to the claims listed in Table~\ref{tab:security-solutions}, which we are going to address in detail in this appendix section. 

First, by ``rootless'' we meant no root privilege is required \emph{after} the initial setup. All solutions listed, including Firejail, requires root privilege to install. However, using it afterwards does not require root privilege.

Second, why in Docker and Podman is considered to be ``rootless'', but we still say that they ``prevent the GPU from being shared by any other container with root access''? It is because both Docker and Podman have a ``rootless mode'', which allows application to run without root access, however it also forces the GPU to run on rootless mode. Take Docker as an example, Docker Engine 20.10 introduced rootless Docker daemon\footnote{\href{https://docs.docker.com/engine/security/rootless/}{https://docs.docker.com/engine/security/rootless/}}. But just like Podman, in rootless mode to make NVidia container runtime work you still need to set no-cgroups = true\footnote{\href{https://www.redhat.com/en/blog/how-use-gpus-containers-bare-metal-rhel-8}{https://www.redhat.com/en/blog/how-use-gpus-containers-bare-metal-rhel-8}}, therefore breaks root access from containers\footnote{\href{https://github.com/NVIDIA/nvidia-container-runtime/issues/85}{https://github.com/NVIDIA/nvidia-container-runtime/issues/85}}.

Third, the ``level of isolation'' of Firejail is labelled as ``medium'' in the table. In fact, Firejail security entirely relies on how strict the security profile is – by default there is no restriction at all. The recommended approach according to Firejail documentation\footnote{\href{https://firejail.wordpress.com/documentation-2/building-custom-profiles/}{https://firejail.wordpress.com/documentation-2/building-custom-profiles/}} is to start with a strictest possible profile, then gradually white-listing features until the application inside the sandbox works fine. Since the full lockdown provided by Firejail is very strict, we consider it as a ``secure-enough'' alternative.

Lastly, VM is labelled to have no GPU support in the table. However, VM does support \textbf{exclusive} GPU access by technologies such as \href{https://docs.fedoraproject.org/en-US/Fedora/13/html/Virtualization_Guide/chap-Virtualization-PCI_passthrough.html}{PCI passthrough}. Exclusivity is the keyword here: we require the solution to work on shared instances, therefore we ignore such infeasible solutions.

\chapter{Proposal for a Multi-agent Tournament System}
\label{appendix:aivle-web_matchmaking}

In the previous version of aiVLE 2.0, we already have infrastructure for running arbitrary evaluation as a \texttt{Job} instance. Here we propose an extension to support multi-agent tournament on aiVLE (both matchmaking and evaluation).

This design is very high-level and abstract: even though all relevant code is concrete and runnable, the matchmaking algorithm/strategy, which happens to be the most important and sophisticated part of tournament systems, is not provided. On the other hand, this also means that our design is not limited to any specific tournament type or matchmaking algorithm.

\section{Tournament Abstraction}
\label{as:matchmaking-api_design-tournament_abstraction}
Before we dive into the actual API design, we need to understand what is common to all tournaments. A tournament consists of one or many rounds of matches. The participants in every match of a certain round are determined before the round starts, and the order of matches within a round does not matter. After every round, we determine whether the tournament has concluded, or generate the match schedule for the next round.

Here we show the generality of such description by applying it to a real-world championship series: the Overwatch League (OWL)\footnote{The Overwatch League is a professional e-sports league for the video game Overwatch. Its structure and operation is similar to other American sports championships such as NBA (National Basketball Association).} championship:

OWL has regular season matches and playoffs. The schedule for regular season is determined before the season starts, and is not affected by the outcome of any matches within. So we may consider the entire regular season as one round.

The list of teams advanced to the playoffs is determined by the regular season results, so starting the playoffs also means starting a new round. Here we take regular season tournament format of OWL 2021 as an example (Figure~\ref{fig:owl-tournament-format}):

\begin{figure}[H]
    \centering
    \includegraphics[width=\textwidth]{images/owl-tournament-format.png}
    \caption{OWL 2021 Tournament Format}
    \label{fig:owl-tournament-format}
\end{figure}

Regional knockouts are single elimination.
\begin{itemize}
    \item Round 1: (west) match 1, 2; (east) match 1, 2
    \item Round 2: (west) match 3, 4
\end{itemize}

Global finals are double elimination.
\begin{itemize}
    \item Round 3: (global) match 1, 2
    \item Round 4: (global) match 3, 4
    \item Round 5: (global) match 5
    \item Round 6: (global) match 6 - the tournament final
\end{itemize}

In fact, this description of tournament is so powerful that when we let every round to include only one match, we can implement general skill-rating algorithms such as Elo Rating~\parencite{elo1978rating}.

\section{API Design}
\label{as:matchmaking-api_design}

\subsection{Overview}
\label{as:matchmaking-api_design-overview}
Now that we have a powerful abstraction of tournaments, we are ready to describe the API design. In specific, a complete tournament consists of the following steps:
\begin{enumerate}
    \item \texttt{GET /api/v1/tasks/<task\_pk>/start\_matchmaking} with parameters:
    \begin{itemize}
        \item List of submission IDs
        \item Matchmaker type (e.g., round-robin)
    \end{itemize}
    \item A \texttt{MatchmakingSession} object is created and stored in DB
    \item \texttt{kickstart} method of the \texttt{MatchmakingSession}'s \texttt{Matchmaker} is called. We get a list of \texttt{MatchParticipant}s and \texttt{Match}es. We store the \texttt{MatchParticipant}s and schedule the \texttt{Match}es just like scheduling single-agent evaluation jobs.
    \item Every call to \texttt{submit\_job} RESTful API will trigger a signal\footnote{For more details, please check \href{https://docs.djangoproject.com/en/4.0/topics/signals/}{https://docs.djangoproject.com/en/4.0/topics/signals/}} to check if the job has a corresponding \texttt{MatchmakingSession}\footnote{For backward compatibility (i.e., supporting evaluation jobs that is not part of any tournament), we allow a \texttt{Job} to have no corresponding \texttt{MatchmakingSession}.}:
    \begin{enumerate}
        \item If yes, check if all \texttt{pending\_jobs} in the \texttt{MatchmakingSession} are finished
        \item If also yes, call \texttt{schedule} method of the \texttt{MatchmakingSession}'s \texttt{Matchmaker} instance
        \begin{enumerate}
            \item Schedule the returned matches by creating \texttt{Job}s if necessary - at the same time reset \texttt{pending\_jobs} of the \texttt{MatchmakingSession}
            \item Update the ratings of the \texttt{MatchParticipant}s using the return value
            \item If concluded, set \texttt{ongoing} of the \texttt{MatchmakingSession} to \texttt{False}
        \end{enumerate}
    \end{enumerate}
\end{enumerate}

\subsection{Models}
There are two new models required for multi-agent tournament: \texttt{MatchParticipant} and \texttt{MatchmakingSession}. In addition, we need a \texttt{Match} wrapper class to represent the list of participants involved in a match. However, unlike the other two models that need to be stored in the database, \texttt{Match} is temporary. The concrete implementation of these models is shown in Code~\ref{code:multi-agent-models}.

\begin{code}
\begin{minted}[frame=lines,framesep=2mm,baselinestretch=1.2,bgcolor=LightGray,fontsize=\footnotesize,linenos,breaklines,samepage]{python}
from django.db import models

class MatchParticipant(models.Model):
    submission = models.ForeignKeyField(Submission)
    rating = models.FloatField()
    

class MatchmakingSession(models.Model):
    participants = models.ManyToManyField(MatchParticipant)
    pending_jobs = models.ManyToManyField(Job)
    ongoing = models.BooleanField()

class Match():
"""
Match is a list of submission IDs corresponding to the participants of this match.
"""
    def __init__(self, ids: List[int]):
        self.ids = ids
\end{minted}
\captionof{listing}{Multi-agent Tournament Models}
\label{code:multi-agent-models}
\end{code}

\subsection{Changes to Existing Models}
\begin{enumerate}
    \item \texttt{Job} model needs to support more than one related \texttt{Submission}s - this can be easily achieved by changing \texttt{ForeignKeyField} to a \texttt{ManyToManyField}.
    \item \texttt{Job} needs to bind to a \texttt{MatchmakingSession} so that when the \texttt{Job} is submitted, the \texttt{Matchmaker} can be notified. One possible way:\\ \texttt{matchmaking\_session = models.ForeignKeyField(MatchmakingSession, null=True)}
\end{enumerate}

\subsection{Matchmaking Logic Abstraction}
\label{as:matchmaking-api_design-matchmaking_logic_abstraction}
As shown in Section~\ref{as:matchmaking-api_design-overview}, the decision-making of the entire matchmaking process is abstracted as the \texttt{kickstart} and \texttt{schedule} method of a \texttt{Matchmaker}. This is exactly where the tournament rules and/or matchmaking algorithm should be implemented. Below is an abstract base class for \texttt{Matchmaker}:

\begin{code}
\begin{minted}[frame=lines,framesep=2mm,baselinestretch=1.2,bgcolor=LightGray,fontsize=\footnotesize,linenos,breaklines,samepage]{python}
class BaseMatchmaker():
"""
Matchmaker class must be stateless - every time a new instance of Matchmaker will be created.
"""
    def kickstart(self, participants: List[MatchParticipant]) -> (List[MatchParticipant], List[Match]):
        pass

    def schedule(self, participants: List[MatchParticipant], jobs: List[Job]) -> (bool, List[MatchParticipant], List[Match]):
        pass
\end{minted}
\captionof{listing}{\texttt{Matchmaker} Abstract Base Class}
\label{code:matchmaker-abc}
\end{code}

Explanation of input parameters:
\begin{itemize}
    \item Pool of participants - in our case, each participant has
    \begin{itemize}
        \item Submission ID
        \item Current rating/score
    \end{itemize}
    \item Match history, in particular, every match should at least have
    \begin{itemize}
        \item IDs of all participating submissions
        \item Outcome of the match
    \end{itemize}
\end{itemize}

Explanation of the outputs:
\begin{itemize}
    \item Whether this tournament has concluded or not
    \item List of updated rating/score for all submissions
    \item List of scheduled matches (if the tournament hasn't concluded yet)
\end{itemize}


% \chapter{Proofs}
% In this appendix, we present alternate, longer, but more interesting proof 
% of correctness of our algorithm.  This proof is based on induction and proof
% by contradiction.
\end{document}
